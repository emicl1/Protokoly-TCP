\documentclass[12pt]{article}
\usepackage[libertine,cmintegrals,cmbraces,vvarbb]{newtxmath}
%\usepackage[T1]{fontenc}
\usepackage[czech]{babel}
\usepackage{hyperref}
\hypersetup{
    colorlinks=true,
    linkcolor=blue,
    filecolor=magenta,      
    urlcolor=cyan,
    pdftitle={Overleaf Example},
    pdfpagemode=FullScreen,
    }
\usepackage{graphicx}
\usepackage[document]{ragged2e}
\graphicspath{ {C:/Users/HP/Desktop/maturitapicture} }
\usepackage{geometry}
 \geometry{
 a4paper,
 total={170mm,257mm},
 left=35mm,
 top=25mm,
bottom=15mm,
right=10mm
 }

\urlstyle{same}


\begin{document}
\begin{center}
\hspace{-2cm}
\includegraphics[scale=3]{logo-dark}


\vspace{4cm}


\hspace{-2.1cm}
\huge\textbf{Maturtní práce}

\vspace{1.5cm}
\hspace{-2cm}
\huge\textbf{Protokoly TCP}
\vspace{0.75cm}
\\
\hspace{-2cm}
\huge\textbf{Alex Olivier Michaud}
\\
\hspace{-2cm}
\vspace{0.75cm}
\large\textbf{Vedoucí práce: Dr.rer.nat. Michal Kočer}
\end{center}

\vspace{6cm}

\rightline\Huge\textbf{V Českých Budějovicích}
\hspace{4cm}
\Large\textbf{Školní rok 2023 } 		


\begin{flushright} 
\end{flushright}
\thispagestyle{empty}





\clearpage
\tableofcontents
\thispagestyle{empty}
\clearpage
\section{Úvod}

\subsection{historie TCP/IP}
V roce 1966 se povedlo v USA Bobu Taylorovi úspěšně sehnat finance od Charles Maria Herzfeld, ředitele ARPA,\footnote{Nyní známo jako DARPA(Defense Advanced Research Projects Agency) je výkonná moc ministerstva obrany Spojených států amerických, které je pověřena vývojem technologií pro vojenské účely} na projekt ARPANET, který měl umožnit přístup k počítačům na velké vzdálenosti. V dalších třech letech se rohodlo o počáteční standardech pro identifikaci, autentizaci uživatelů, přenos znaků a kontrolu a roku 1969 byl ARPANET poprvé použit firmou BBN. 
Při dalším výzkmu a pokusech o vytvoření nového modulu ARPANET, dva vědci Robert Elliot Kahn a Vinton Gray Cerf vytvořili nový model, kde hlavní zodpovědnost za spolehlivost byla předána uživateli místo sítě. Tímto roku 1974 vznikl nový protokol Transmission Control Program, který byl vydán v RFC\footnote{žádost o komentáře - označuje dokumenty popisující internetové protkoly} 675 s názvem Specification of Internet Transmission Control Program, avšak tato verze nebyla funkční až do roku 1981, kdy byla zprovozněna verzí 4. Je standardizována pomocí RFC 791 - Internet Protocol(IP) a RFC 793 Transmission Control Protocol(TCP). 
\\
TCP i IP, prošlo s postupem času velkým vývojem, kdy vznikalo stovky aktualizací. Například roku 1994 vzniklo Internet Protocol next generation (IPng), který zavadí IP verzi 6. Nyní se aktivně používá 10+ variant TCP na Linuxu. MacOS a Windows je má zavedeno jako výchozí nastavení. 

\subsection{Základy komunikace aplikací na úrovni TCP/IP}

\subsection{Link layer}

\subsection{Internet layer}

\subsection{Transport layer}

\subsection{Application layer}


\clearpage
\section{zdroje}
\subsection{historie tcp/ip}
\url{https://www.geeksforgeeks.org/history-of-tcp-ip}
\\
\url{https://scos.training/history-of-tcp-ip}
\\
\url{https://en.wikipedia.org/wiki/ARPANET}
\\
\url{https://en.wikipedia.org/wiki/Request_for_Comments}
\\
\url{https://en.wikipedia.org/wiki/DARPA}
\\
\url{https://cs.wikipedia.org/wiki/Request_for_Comments}
\\
\url{}
\\
\url{}
\\
\subsection{základy komunikace aplikací na úrovni TCP/IP}
\url{}
\\
\url{}
\\
\url{}
\\
\url{}
\\
\url{}
\\
\url{}
\\
\url{}
\\
\subsection{Link layer}
\url{}
\subsection{Internet layer}
\url{}
\subsection{Transport layer}
\url{}
\subsection{Application layer}
\url{}




\end{document}









